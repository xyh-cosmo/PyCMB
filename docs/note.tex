\documentclass[a4paper,10pt]{ctexart}
%\usepackage{xeCJK}

\usepackage[text={6.5in,9in},centering]{geometry}
\usepackage{xcolor}
\usepackage{amsmath}
\usepackage{setspace}
\usepackage{bookmark}

\begin{document}

\title{PyCMB--A \textbf{P}ython code for \textbf{C}osmic \textbf{M}ircowave \textbf{B}ackground
Radiation Power Spectrum Calculation}
\author{Youhua Xu}
\maketitle

\abstract{紫微,这是为你而写的一个文档.}

\tableofcontents

%设置行距
%\begin{spacing}{1.25}

\section{相关知识准备}

\subsection{Python基础知识}

\subsection{基础数值算法的Python实现}

\subsection{天文学和宇宙学中常见的计算}


\section{化整为零}

\subsection{宇宙的背景演化}

\subsection{宇宙电离度的演化}

\subsection{扰动的演化}


\section{最后一步:化零为整}


\appendix
\section{编程建议}



%\end{spacing}
\end{document}
